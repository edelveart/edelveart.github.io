\documentclass{article}
\usepackage{cheatsheet-style}

\title{\huge \textbf{FigurateNum} \\
   \Large  A Python library for generating infinite figurate number sequences across dimensions.}
\author{}
\date{}

\begin{document}

\begin{tcolorbox}[leftrule=1mm, rightrule=1mm, toprule=1mm, bottomrule=1mm ]
    \maketitle
    \vspace{-5em}
    \begin{multicols}{2}

        \begin{tcolorbox}[leftrule=2mm, rightrule=0mm, toprule=0mm, bottomrule=0mm,  colframe=SpringGreen4, colback=SpringGreen4!12, height=2cm]
            \textbf{\large Installation}
            \begin{lstlisting}
    pip install figuratenum
    pip install figuratenum[figurate-viz] # Optional: Visualization (v2.1.0)
            \end{lstlisting}
        \end{tcolorbox}

        \begin{tcolorbox}[leftrule=2mm, rightrule=0mm, toprule=0mm, bottomrule=0mm,  colframe=SpringGreen4, colback=SpringGreen4!12, height=1.7cm]
            \textbf{\large Main Class}
            \begin{lstlisting}
    from figuratenum import FigurateNum as fgn
             \end{lstlisting}
        \end{tcolorbox}

        \begin{tcolorbox}[leftrule=2mm, rightrule=0mm, toprule=0mm, bottomrule=0mm,  colframe=SpringGreen4, colback=SpringGreen4!12, height=3.6cm]
            \textbf{\large Classes}
            \begin{lstlisting}
    from figuratenum import PlaneFigurateNum as pfgn
    from figuratenum import SpaceFigurateNum as sfgn
    from figuratenum import MultidimensionalFigurateNum as mfgn
    from figuratenum import ZooFigurateNum as zfgn
    from figuratenum import NumCollector as nc
    from figuratenum.figurate_viz.FigurateViz import FigurateViz
             \end{lstlisting}
        \end{tcolorbox}

    \end{multicols}
\end{tcolorbox}

\thispagestyle{fancy}

\begin{multicols}{2}
    \begin{tcolorbox}[title= Import and generate sequences with FigurateNum and related classes]
        \begin{lstlisting}
    from figuratenum import FigurateNum, MultidimensionalFigurateNum
    # 1. General use: generate any figurate sequence via FigurateNum
    seq = FigurateNum()
    hyperdodecahedral_gen = seq.hyperdodecahedral()
    print([next(hyperdodecahedral_gen) for _ in range(4)])
    # Output: [1, 600, 4983, 19468]
    # 2. Specialized classes: PlaneFigurateNum, SpaceFigurateNum,
    #    MultidimensionalFigurateNum, ZooFigurateNum
    multi = MultidimensionalFigurateNum()
    hypertetrahedron_gen = multi.k_dimensional_centered_hypertetrahedron(21)
    print([next(hypertetrahedron_gen) for _ in range(10)])
    # Output: [1, 23, 276, 2300, 14950, 80730, 376740, 1560780, 5852925, 20160075]
        \end{lstlisting}
    \end{tcolorbox}

    \begin{tcolorbox}[title = Using FigurateViz to visualize and export]
        \begin{lstlisting}
    from figuratenum import FigurateNum as fgn
    from figuratenum.figurate_viz.FigurateViz import FigurateViz
    seq_loop = fgn()
    gen = seq_loop.five_dimensional_hyperoctahedron()
    figuratenum_seq = [next(gen) for _ in range(704)]
    # Create and draw the Gaussian plot
    viz = FigurateViz(figuratenum_seq, figsize=(6, 6))
    viz.gaussian_plot(circ_color="m", bg_color="k", num_text=False,
        num_color="g", ext_circle=True, rotate=-1).draw()
    # Export plots as .svg, .pdf, .png (matplotlib compatible),
    # with options e.g., dpi, transparent, bbox_inches, pad_inches, etc.
    viz.export_plot("figure1.svg", circ_color="cyan", transparent=True)
        \end{lstlisting}
    \end{tcolorbox}

\end{multicols}

\begin{multicols}{2}
    \begin{tcolorbox}[title= NumberCollector Class: Method Summary]
        \begin{itemize}[noitemsep, topsep=0pt]
            \item \bera{take(n)}
            \item \bera{take\_to\_list(stop, start, step)}
            \item \bera{take\_to\_array(stop, start, step)}
            \item \bera{take\_to\_tuple(stop, start, step)}
            \item \bera{pick(n)}
        \end{itemize}
    \end{tcolorbox}

    \begin{tcolorbox}[title= Working with the NumberCollector Class]
        \begin{lstlisting}
    from figuratenum import NumCollector as nc, FigurateNum
    gen = FigurateNum().pentatope()
    print(nc.take_to_tuple(gen, 10))  # first 10 values as tuple
    # Output: (1, 5, 15, 35, 70, 126, 210, 330, 495, 715)
        \end{lstlisting}
    \end{tcolorbox}
\end{multicols}

\pagebreak

\begin{multicols}{2}
    \begin{tcolorbox}[title= Plane Figurate Numbers]
        \begin{itemize}[noitemsep, topsep=0pt]
            \item \bera{polygonal}
            \item \bera{triangular}
            \item \bera{square}
            \item \bera{pentagonal}
            \item \bera{hexagonal}
            \item \bera{heptagonal}
            \item \bera{octagonal}
            \item \bera{nonagonal}
            \item \bera{decagonal}
            \item \bera{hendecagonal}
            \item \bera{dodecagonal}
            \item \bera{tridecagonal}
            \item \bera{tetradecagonal}
            \item \bera{pentadecagonal}
            \item \bera{hexadecagonal}
            \item \bera{heptadecagonal}
            \item \bera{octadecagonal}
            \item \bera{nonadecagonal}
            \item \bera{icosagonal}
            \item \bera{icosihenagonal}
            \item \bera{icosidigonal}
            \item \bera{icositrigonal}
            \item \bera{icositetragonal}
            \item \bera{icosipentagonal}
            \item \bera{icosihexagonal}
            \item \bera{icosiheptagonal}
            \item \bera{icosioctagonal}
            \item \bera{icosinonagonal}
            \item \bera{triacontagonal}
            \item \bera{centered\_triangular}
            \item \bera{centered\_square = diamond}
            \item \bera{centered\_pentagonal}
            \item \bera{centered\_hexagonal}
            \item \bera{centered\_heptagonal}
            \item \bera{centered\_octagonal}
            \item \bera{centered\_nonagonal}
            \item \bera{centered\_decagonal}
            \item \bera{centered\_hendecagonal}
            \item \bera{centered\_dodecagonal = star}
            \item \bera{centered\_tridecagonal}
        \end{itemize}
    \end{tcolorbox}

    \begin{tcolorbox}[title= Plane Figurate Numbers]
        \begin{itemize}[noitemsep, topsep=0pt]
            \item \bera{centered\_tetradecagonal}
            \item \bera{centered\_pentadecagonal}
            \item \bera{centered\_hexadecagonal}
            \item \bera{centered\_heptadecagonal}
            \item \bera{centered\_octadecagonal}
            \item \bera{centered\_nonadecagonal}
            \item \bera{centered\_icosagonal}
            \item \bera{centered\_icosihenagonal}
            \item \bera{centered\_icosidigonal}
            \item \bera{centered\_icositrigonal}
            \item \bera{centered\_icositetragonal}
            \item \bera{centered\_icosipentagonal}
            \item \bera{centered\_icosihexagonal}
            \item \bera{centered\_icosiheptagonal}
            \item \bera{centered\_icosioctagonal}
            \item \bera{centered\_icosinonagonal}
            \item \bera{centered\_triacontagonal}
            \item \bera{centered\_mgonal(m)}
            \item \bera{pronic = heteromecic = oblong}
            \item \bera{polite}
            \item \bera{impolite}
            \item \bera{cross}
            \item \bera{aztec\_diamond}
            \item \bera{polygram(m) = centered\_star\_polygonal(m)}
            \item \bera{pentagram}
            \item \bera{gnomic}
            \item \bera{truncated\_triangular}
            \item \bera{truncated\_square}
            \item \bera{truncated\_pronic}
            \item \bera{truncated\_centered\_pol(m) = truncated\_centered\_mgonal(m)}
            \item \bera{truncated\_centered\_triangular}
            \item \bera{truncated\_centered\_square}
            \item \bera{truncated\_centered\_pentagonal}
            \item \bera{truncated\_centered\_hexagonal = truncated\_hex}
            \item \bera{generalized\_mgonal(m, start\_numb)}
            \item \bera{generalized\_pentagonal(start\_numb)}
            \item \bera{generalized\_hexagonal(start\_numb)}
            \item \bera{generalized\_centered\_pol(m, start\_numb)}
            \item \bera{generalized\_pronic(start\_numb)}
        \end{itemize}
    \end{tcolorbox}
\end{multicols}

\pagebreak
\begin{multicols}{3}
    \begin{tcolorbox}[title= Space Figurate Numbers]
        \begin{itemize}[noitemsep, topsep=0pt]
            \item \bera{m\_pyramidal(m)}
            \item \bera{triangular\_pyramidal}
            \item \bera{square\_pyramidal = pyramidal}
            \item \bera{pentagonal\_pyramidal}
            \item \bera{hexagonal\_pyramidal}
            \item \bera{heptagonal\_pyramidal}
            \item \bera{octagonal\_pyramidal}
            \item \bera{nonagonal\_pyramidal}
            \item \bera{decagonal\_pyramidal}
            \item \bera{hendecagonal\_pyramidal}
            \item \bera{dodecagonal\_pyramidal}
            \item \bera{tridecagonal\_pyramidal}
            \item \bera{tetradecagonal\_pyramidal}
            \item \bera{pentadecagonal\_pyramidal}
            \item \bera{hexadecagonal\_pyramidal}
            \item \bera{heptadecagonal\_pyramidal}
            \item \bera{octadecagonal\_pyramidal}
            \item \bera{nonadecagonal\_pyramidal}
            \item \bera{icosagonal\_pyramidal}
            \item \bera{icosihenagonal\_pyramidal}
            \item \bera{icosidigonal\_pyramidal}
            \item \bera{icositrigonal\_pyramidal}
            \item \bera{icositetragonal\_pyramidal}
            \item \bera{icosipentagonal\_pyramidal}
            \item \bera{icosihexagonal\_pyramidal}
            \item \bera{icosiheptagonal\_pyramidal}
            \item \bera{icosioctagonal\_pyramidal}
            \item \bera{icosinonagonal\_pyramidal}
            \item \bera{triacontagonal\_pyramidal}
            \item \bera{triangular\_tetrahedral[finite]}
            \item \bera{triangular\_square\_pyramidal[finite]}
            \item \bera{square\_tetrahedral[finite]}
            \item \bera{square\_square\_pyramidal[finite]}
            \item \bera{tetrahedral\_square\_pyramidal[finite]}
            \item \bera{cubic}
            \item \bera{tetrahedral}
            \item \bera{octahedral}
            \item \bera{dodecahedral}
            \item \bera{icosahedral}
            \item \bera{truncated\_tetrahedral}
        \end{itemize}
    \end{tcolorbox}

    \begin{tcolorbox}[title= Space Figurate Numbers]
        \begin{itemize}[noitemsep, topsep=0pt]
            \item \bera{truncated\_cubic}
            \item \bera{truncated\_octahedral}
            \item \bera{stella\_octangula}
            \item \bera{centered\_cube}
            \item \bera{rhombic\_dodecahedral}
            \item \bera{hauy\_rhombic\_dodecahedral}
            \item \bera{centered\_tetrahedron = centered\_tetrahedral}
            \item \bera{centered\_square\_pyramid = centered\_pyramid}
            \item \bera{centered\_mgonal\_pyramid(m)}
            \item \bera{centered\_pentagonal\_pyramid}
            \item \bera{centered\_hexagonal\_pyramid}
            \item \bera{centered\_heptagonal\_pyramid}
            \item \bera{centered\_octagonal\_pyramid}
            \item \bera{centered\_octahedron}
            \item \bera{centered\_icosahedron = centered\_cuboctahedron}
            \item \bera{centered\_dodecahedron}
            \item \bera{centered\_truncated\_tetrahedron}
            \item \bera{centered\_truncated\_cube}
            \item \bera{centered\_truncated\_octahedron}
            \item \bera{centered\_mgonal\_pyramidal(m)}
            \item \bera{centered\_triangular\_pyramidal}
            \item \bera{centered\_square\_pyramidal}
            \item \bera{centered\_pentagonal\_pyramidal}
            \item \bera{centered\_heptagonal\_pyramidal}
            \item \bera{centered\_octagonal\_pyramidal}
            \item \bera{centered\_nonagonal\_pyramidal}
            \item \bera{centered\_decagonal\_pyramidal}
            \item \bera{centered\_hendecagonal\_pyramidal}
            \item \bera{centered\_dodecagonal\_pyramidal}
            \item \bera{centered\_hexagonal\_pyramidal = hex\_pyramidal}
            \item \bera{hexagonal\_prism}
            \item \bera{mgonal\_prism(m)}
            \item \bera{generalized\_mgonal\_pyramidal(m, start\_num)}
            \item \bera{generalized\_pentagonal\_pyramidal(start\_num)}
            \item \bera{generalized\_hexagonal\_pyramidal(start\_num)}
            \item \bera{generalized\_cubic(start\_num)}
            \item \bera{generalized\_octahedral(start\_num)}
        \end{itemize}
    \end{tcolorbox}

    \begin{tcolorbox}[title= Space Figurate Numbers]
        \begin{itemize}[noitemsep, topsep=0pt]
            \item \bera{generalized\_icosahedral(start\_num)}
            \item \bera{generalized\_dodecahedral(start\_num)}
            \item \bera{generalized\_centered\_cube(start\_num)}
            \item \bera{generalized\_centered\_tetrahedron(start\_num)}
            \item \bera{generalized\_centered\_square\_pyramid(start\_num)}
            \item \bera{generalized\_rhombic\_dodecahedral(start\_num)}
            \item \bera{generalized\_centered\_mgonal\_pyramidal(m, start\_num)}
            \item \bera{generalized\_mgonal\_prism(m, start\_num)}
            \item \bera{generalized\_hexagonal\_prism(start\_num)}
        \end{itemize}
    \end{tcolorbox}
\end{multicols}

\begin{multicols}{2}
    \begin{tcolorbox}[bottomrule=0mm, title= Multidimensional Figurate Numbers]
        \begin{itemize}[noitemsep, topsep=0pt]
            \item \bera{k\_dimensional\_hypertetrahedron(k) = k\_hypertetrahedron(k) = regular\_k\_polytopic(k) = figurate\_of\_order\_k(k)}
            \item \bera{five\_dimensional\_hypertetrahedron}
            \item \bera{six\_dimensional\_hypertetrahedron}
            \item \bera{k\_dimensional\_hypercube(k) = k\_hypercube(k)}
            \item \bera{five\_dimensional\_hypercube}
            \item \bera{six\_dimensional\_hypercube}
            \item \bera{hypertetrahedral = pentachoron = pentatope = triangulotriangular = cell\_5}
            \item \bera{hypercube = octachoron = tesseract = biquadratic = cell\_8 }
            \item \bera{hyperoctahedral = hexadecachoron = four\_cross\_polytope = four\_orthoplex = cell\_16}
            \item \bera{hypericosahedral = hexacosichoron = polytetrahedron = tetraplex = cell\_600}
            \item \bera{hyperdodecahedral = hecatonicosachoron = dodecaplex = polydodecahedron = cell\_120}
            \item \bera{polyoctahedral = icositetrachoron = octaplex = hyperdiamond = cell\_24}
            \item \bera{four\_dimensional\_hyperoctahedron}
            \item \bera{five\_dimensional\_hyperoctahedron}
            \item \bera{six\_dimensional\_hyperoctahedron}
            \item \bera{seven\_dimensional\_hyperoctahedron}
            \item \bera{eight\_dimensional\_hyperoctahedron}
            \item \bera{nine\_dimensional\_hyperoctahedron}
            \item \bera{ten\_dimensional\_hyperoctahedron}
            \item \bera{k\_dimensional\_hyperoctahedron(k) = k\_cross\_polytope(k)}
            \item \bera{four\_dimensional\_mgonal\_pyramidal(m) = mgonal\_pyramidal\_of\_the\_second\_order(m)}
            \item \bera{four\_dimensional\_square\_pyramidal}
            \item \bera{four\_dimensional\_pentagonal\_pyramidal}
            \item \bera{four\_dimensional\_hexagonal\_pyramidal}
            \item \bera{four\_dimensional\_heptagonal\_pyramidal}
            \item \bera{four\_dimensional\_octagonal\_pyramidal}
            \item \bera{four\_dimensional\_nonagonal\_pyramidal}
            \item \bera{four\_dimensional\_decagonal\_pyramidal}
            \item \bera{four\_dimensional\_hendecagonal\_pyramidal}
            \item \bera{four\_dimensional\_dodecagonal\_pyramidal}
            \item \bera{k\_dimensional\_mgonal\_pyramidal(k,m) = mgonal\_pyramidal\_of\_the\_k\_2\_th\_order(k,m)}
            \item \bera{five\_dimensional\_mgonal\_pyramidal(m)}
            \item \bera{five\_dimensional\_square\_pyramidal}
            \item \bera{five\_dimensional\_pentagonal\_pyramidal}
            \item \bera{five\_dimensional\_hexagonal\_pyramidal}
            \item \bera{five\_dimensional\_heptagonal\_pyramidal}
            \item \bera{five\_dimensional\_octagonal\_pyramidal}
        \end{itemize}
    \end{tcolorbox}

    \begin{tcolorbox}[bottomrule=0mm, title= Multidimensional Figurate Numbers]
        \begin{itemize}[noitemsep, topsep=0pt]

            \item \bera{six\_dimensional\_mgonal\_pyramidal(m)}
            \item \bera{six\_dimensional\_square\_pyramidal}
            \item \bera{six\_dimensional\_pentagonal\_pyramidal}
            \item \bera{six\_dimensional\_hexagonal\_pyramidal}
            \item \bera{six\_dimensional\_heptagonal\_pyramidal}
            \item \bera{six\_dimensional\_octagonal\_pyramidal}
            \item \bera{centered\_biquadratic}
            \item \bera{k\_dimensional\_centered\_hypercube(k)}
            \item \bera{five\_dimensional\_centered\_hypercube}
            \item \bera{six\_dimensional\_centered\_hypercube}
            \item \bera{centered\_polytope}
            \item \bera{k\_dimensional\_centered\_hypertetrahedron(k)}
            \item \bera{five\_dimensional\_centered\_hypertetrahedron}
            \item \bera{six\_dimensional\_centered\_hypertetrahedron}
            \item \bera{centered\_hyperoctahedral}
            \item \bera{nexus(k)}
            \item \bera{k\_dimensional\_centered\_hyperoctahedron(k)}
            \item \bera{five\_dimensional\_centered\_hyperoctahedron}
            \item \bera{six\_dimensional\_centered\_hyperoctahedron}
            \item \bera{generalized\_pentatope(start\_num = 0)}
            \item \bera{generalized\_k\_dimensional\_hypertetrahedron(k = 5, start\_num = 0)}
            \item \bera{generalized\_biquadratic(start\_num = 0)}
            \item \bera{generalized\_k\_dimensional\_hypercube(k = 5, start\_num = 0)}
            \item \bera{generalized\_hyperoctahedral(start\_num = 0)}
            \item \bera{generalized\_k\_dimensional\_hyperoctahedron(k = 5, start\_num = 0)}
            \item \bera{generalized\_hyperdodecahedral(start\_num = 0)}
            \item \bera{generalized\_hypericosahedral(start\_num = 0)}
            \item \bera{generalized\_polyoctahedral(start\_num = 0)}
            \item \bera{generalized\_k\_dimensional\_mgonal\_pyramidal(k, m, start\_num = 0)}
            \item \bera{generalized\_k\_dimensional\_centered\_hypercube(k, start\_num = 0)}
            \item \bera{generalized\_nexus(k, start\_num = 0)}
        \end{itemize}
    \end{tcolorbox}

\end{multicols}

\begin{multicols}{2}
    \begin{tcolorbox}[bottomrule=0mm, title= Zoo Figurate Numbers]
        \begin{itemize}[noitemsep, topsep=0pt]
            \item \bera{cuban\_prime}
            \item \bera{pell}
        \end{itemize}
    \end{tcolorbox}
\end{multicols}

\end{document}


