
%Change the font size of your document - 10pt, 12.1pt, etc.
\documentclass[letterpaper,10pt,oneside]{article}
\usepackage[utf8]{inputenc}
\usepackage[T1]{fontenc}
\usepackage{setspace}
\usepackage{hyperref}
\usepackage{lmodern}
\usepackage{xcolor}

\usepackage{graphicx}
% \graphicspath{ {images/}} %upload your signature to this file
%Change the margins to fit your CV/resume content
\usepackage[left=0.8in, right=0.8in, bottom=0.5in, top=0.8in]{geometry}

%Skype information - include your Skype name for a link to add you on Skype
% \newcommand*{\Skype}{\href{skype:john.smith?add}{john.smith}}
% \newcommand{\Absender}[1][\normalsize]{\Skype}

%Changes the page numbers - {arabic}=arabic numerals, {gobble}=no page numbers, {roman}=Roman numerals
\pagenumbering{gobble}


%%%%%%%%%%%%%%%%% END OF PREAMBLE %%%%%%%%%%%%%%%%%%%%%
\begin{document}

%%%%%%%%%%%%%%%%% NAME OF APPLICANT %%%%%%%%%%%%%%%%%%%

% \noindent
% \huge{\textbf{Edgar Delgado Vega}}
% \normalsize{
%   \hspace{.99in} \href{https://www.linkedin.com/in/edgararmandodelgadovega/}{\scshape{linkedin}}
%   \hspace{0.1in} \href{https://edelveart.github.io/}{\scshape{web page}}
%   \hspace{0.1in} \href{https://github.com/edelveart}{\scshape{github}}
%   \hspace{0.1in} \href{https://www.youtube.com/@edelvemusic}{\scshape{youtube}
% }
% \vspace{0.75em} \\
% \large{{Desarrollador de software y matemático entusiasta}}

% \vspace{0.5em}

% \hrule

\noindent
\huge{\bf{Edgar Delgado Vega}}
\vspace{0.25em} \\
\large{{{Desarrollador de software y entusiasta de las matemáticas}}}


\vspace{1.25em}

\normalsize{
  \noindent
  \textcolor{teal}{{\scshape{linkedin.com/in/edgararmandodelgadovega}} $\ \  \cdot$
    \hspace{0.1in} {\scshape{edelveart.github.io}} $\ \  \cdot$
    \hspace{0.1in} {\scshape{github.com/edelveart}}
    % \hspace{0.1in} \href{https://www.youtube.com/@edelvemusic}{\scshape{youtube}}
    }
    }
% \normalsize{
%   \noindent
%   \href{https://www.linkedin.com/in/edgararmandodelgadovega/}{\scshape{linkedin.com/in/edgararmandodelgadovega}}
%   \hspace{0.1in} \href{https://edelveart.github.io/}{\scshape{edelveart.github.io}}
%   \hspace{0.1in} \href{https://github.com/edelveart}{\scshape{github.com/edelveart}}
%   % \hspace{0.1in} \href{https://www.youtube.com/@edelvemusic}{\scshape{youtube}}
% }

  \textcolor{teal}{\hrule}

  %%%%%%%%%%%%%%%%% CONTACT INFORMATION %%%%%%%%%%%%%%%%%
% Your email address, website, and Skype name are links to send email, open your website and add you on Skype.

% \begin{center}
% \begin{tabular}{l l}
%  Name of University    & \hspace{1in} \href{mailto:john.smith@email.com}{john.smith@email.com} \\
%  Department/Institution    & \hspace{1in}  \href{www.johnsmith.com}{www.johnsmith.com}   \\
%  Address             & \hspace{1in} Skype: \Absender  \\
%  City Name, State 12345-6789 & \hspace{1in} Phone: +1 (123) 456-7899 \\
% \end{tabular}
% \end{center}
% \begin{center}
% \begin{tabular}{l l}
%  & \hspace{1in}  \href{https://edelveart.github.io/}{Personal web page}   \\
%  & \hspace{1in}  \href{https://github.com/edelveart}{GitHub}   \\
%  & \hspace{1in}  \href{https://www.linkedin.com/in/edgararmandodelgadovega/}{LinkedIn}   \\
%  & \hspace{1in}  \href{https://www.youtube.com/@edelvemusic}{YouTube}   \\
% \end{tabular}
% \end{center}


% \noindent
% \hspace{1.25in} \href{https://www.linkedin.com/in/edgararmandodelgadovega/}{LinkedIn}
%   \hspace{0.25in} \href{https://edelveart.github.io/}{Personal web page}
%   \hspace{0.25in} \href{https://github.com/edelveart}{GitHub}
%   \hspace{0.25in} \href{https://www.youtube.com/@edelvemusic}{YouTube}


\vspace{2em}

%%%%%%%%%%%%%%%%% MAIN BODY %%%%%%%%%%%%%%%%%%%%%%%%%%%
% The main body is contained in a tabular environment. To move sections onto the next page, simply end the tabular environment and begin a new tabular environment.
\noindent Soy desarrollador TypeScript con conocimiento en algoritmos numéricos y patrones de diseño. Escribí el artículo \href{https://link.springer.com/chapter/10.1007/978-3-031-07015-0_30}{Formal Structures of a Harmony in the Parabola}, indexado en Scopus. En 2023, comencé a desarrollar ts-tonnetz, un paquete para composición algorítmica usado por el entorno de codificación \href{https://topos.live/#ziffers_tonnetz}{Topos}.

Además, mi entusiasmo por las matemáticas se refleja en mi gema figurate\_numbers, que ha superado las 2000 descargas. También colaboro internacionalmente en la creación de software libre.
\vspace{2em}

\noindent
\begin{tabular}{@{}l l}

  \textcolor{teal}{\rule{0.4em}{7pt}}
  \scshape{\textcolor{teal}{Experiencia}}
  & \textbf{ZifferJS | TypeScript Software Developer} \\
  &  oct.2023 - actualidad | Finlandia | En remoto \\
  & Colaboro en el desarrollo de una nueva notación generativa y analizador \\& en TypeScript, diseñado para entornos de live coding y algoritmos.  \\ &  Mi responsabilidad es crear y ampliar el módulo Tonnetz, estableciéndolo \\ &  como el recurso más completo y avanzado de su campo en la actualidad.\\&
  Además, me encargo de redactar la documentación del proyecto \\& y de llevar a cabo pruebas unitarias utilizando Vitest.\\
  & \\

  & \textbf{Escuela de Posgrado Newman | Docente de Maestría en Musicología} \\
    &  may.2024 - actualidad | Lima, Perú | En remoto \\
    & Responsable de las asignaturas: \\
    & $\cdot$ Técnicas de búsqueda, documentación y difusión en investigación musicológica.  \\
    & $\cdot$ Industrias musicales. \\
    & \\


    % &\textbf{EEST Toulouse Lautrec | Docente de producción musical}  \\
    % & jun.2022 - actualidad | Lima, Perú | En remoto \\
    % & Responsable de los cursos de producción musical y teoría musical. \\
    % & Los comentarios sobre mi práctica docente son mayoritariamente favorables. \\
    % & \\

    \textcolor{teal}{\rule{0.4em}{7pt}}
    \scshape{\textcolor{teal}{Proyectos}}
    & \textbf{ts-tonnetz} \\
    & Biblioteca en desarrollo para la composición algorítmica e improvisación, \\ & basada en dos tesis doctorales de musicología computacional y matemática \\ & (Bigo, 2013; Cannas, 2018) y en un conjunto de artículos  de los últimos años \\ & en teoría transformacional. Disponible en npm: \href{https://www.npmjs.com/package/ts-tonnetz }{{ts-tonnetz}}. \\

    & \\
    & \textbf{figurate\_numbers} \\
    & Módulo de Ruby que genera 241 secuencias infinitas  de números figurados \\ &multidimensionales del libro "Figurate Numbers" (2012), escrito por Elena Deza \\ &  y Michel Deza.  Está pensado para usarse en proyectos matemáticos y en Sonic Pi.  \\ & Disponible en RubyGems: \href{https://rubygems.org/gems/figurate_numbers }{figurate\_numbers}.\\
    & \\


    % \scshape{Idiomas}   & Inglés (básico), Español (nativo) \\

    \textcolor{teal}{\rule{0.4em}{7pt}}
    \scshape{\textcolor{teal}{Tecnologías}}

    % & \textbf{Desarrollador TypeScript} \\
    % & TypeScript, Node.js, Express,  SQL, Git, HTML5, CSS, Bootstrap  \\
    % & \\
    & \textbf{Desarrollador front-end TypeScript} \\
    & TypeScript, Git, HTML5, CSS, Bootstrap, Tailwind  \\
    & \\

    % & \textbf{Otras habilidades tecnológicas} \\
    % & Sonic Pi, MuseScore, \LaTeX, Reaper, Geogebra, \\
    % & cierto conocimiento de Ruby y Python. \\
    % & \\

    % \textcolor{teal}{\rule{0.4em}{7pt}}
    % \scshape{\textcolor{teal}{Líneas de interés}}
    %  & \textbf{Matemática, algoritmos y música} \\
    % & Teoría matemática de la música. \\
    % & Lenguajes específicos de dominio (DSL) para la música. \\
    % & Composición e improvisación musical asistida por ordenador.\\
    % & \\

    \textcolor{teal}{\rule{0.4em}{7pt}}
    \scshape{\textcolor{teal}{Educación}}
    & \textbf{UNIR |  Máster en Investigación Musical} \\
     & 2020 | España \\
     & Matrícula de Honor en Análisis Musical Informatizado.  \\
     & Calificación general sobresaliente: 9.4. \\
     & \\
    %  & \textbf{USMP | Licenciatura en Música} \\
    %  &  2015 | Lima, Perú  \\
    %  & \\


%  \scshape{Experiencia}    & ``Title of Dissertation" \\
%     & \parbox{5.0in}{Short description/summary of research and estimation techniques. This can be several lines long because of the paragraph box.}\\
%     & \\





\end{tabular}

\newpage


% %%%%%%%%%%%%%%%%% REFERENCES %%%%%%%%%%%%%%%%%%%%%%%%%%
% % The reference section has links to your references' websites and email addresses.

% \noindent \begin{tabular}{@{} l l l}
%  \Large{References} & \href{http://www.professorone.com}{Professor One} & \href{http://www.professortwo.com}{Professor Two} \\
%  & Department Name &  Department Name  \\
%  & University Name &  University Name \\
%  & \small{\href{mailto:prof1@email.com}{prof1@email.com},+1\,(123)\,456-7899} & \small{\href{mailto:prof2@email.com}{prof2@email.com},+1\,(987)\,654-3210} \\
% && \\
%  & \href{http://www.professorthree.com}{Professor Three} & \href{http://www.professorfour.com}{Professor Four}  \\
%  & Department Name &  Department Name \\
%  & University Name &  University Name \\
%  & \small{\href{mailto:prof3@email.com}{prof3@email.com},+1\,(123)\,789-1011} & \small{\href{mailto:prof4@email.com}{prof4@email.com},+1\,(789)\,456-9879} \\
% \end{tabular}



% \clearpage
% \setlength\parindent{0cm}
% \pagenumbering{gobble} %cover letter should be one page, {gobble}=no page number



% \begin{flushright}
%  \today                           \\
%  \vspace{1em}
%  Home University            \\
%  Home Department                  \\
%  Street Address                       \\
%  City, State. 12345-67899   \\
%  Phone: +1 (123) 456-7899         \\
% \href{mailto:john.smith@email.com}{john.smith@email.com}  \\ %insert your email address here for a clickable link
% \end{flushright}


% \begin{flushleft}
%  \textbf{Faculty Search Committee}         \\
%  Name of University \\
% Name of Department                  \\
% Address of Department \\
% City, State. Zip Code
% \end{flushleft}

% \vspace{2em}

% Dear Sir or Madam, \\

% \vspace{1em}
% \onehalfspacing

% My name is John Smith and I'm applying to the academic position in this subject at the Name of University. I have experience teaching something and something else and my research focuses on this and that. I completed my Ph.D. in this subject in September 2015 at my alma mater.

% \vspace{1em}

% My educational background is in this and that at the former university along with an earned Master's and Ph.D. degree in this recent subject at this alma mater. I have taught this, that, and everything else. My research is in this area, that area, and another area still.

% \vspace{1em}

% My dissertation advisor, Professor Head Adviser, and committee members Professor Two and Professor Three have been instrumental throughout my time at Home University. Please feel free to contact me with any questions.

% \vspace{1em}

% \begin{flushright}
% Sincerely, \\
% \vspace{1em}
% % \includegraphics[scale=0.4]{Phillips} \\ %insert your own signature here
% \vspace{1em}
% John Smith \\
% \end{flushright}

\end{document}
